%! Licence = CC BY-NC-SA 4.0

%! Author = mariuszindel
%! Date = 22. Feb 2021
%! Project = latex-documentation-template

\section{Section}



\subsection{Subsection}


\subsubsection{Subsubsection}
\lipsum[1]



\subsubsection{graphic}
\begin{figure}[h]
    \centering
    \includegraphics[scale=0.5]{img/OST_Logo}
    \caption{The OST Logo!}
\end{figure}

\subsubsection{table}
\begin{table}[h]
    \centering
    \caption{example tabular}
    \label{tab:example tabular}
        \begin{tabular}{l | c | r}
            \hline
            left align & center align & right align\\ \hline
            \hline
            hello & world & !\\ \hline
            hello & world & !\\ \hline
        \end{tabular}
\end{table}

\subsubsection{items}
\begin{itemize}
    \item hello
    \item world
    \item !
\end{itemize}

\subsubsection{enumaeration}
\begin{enumerate}
    \item hello
    \item world
    \item !
\end{enumerate}

\subsubsection{listings}
\begin{lstlisting}[caption=example listing]
//simple listing in default language
    for (int i = 0; i < 10; i++) {
    System.out.printline(i + "hello world!");
    }
\end{lstlisting}

\begin{lstlisting}[firstnumber=17, caption=example listing, style=JavaScript]
// example listing in JS
    var = hello world
\end{lstlisting}


\subsubsection{minipage}
\begin{minipage}{0.5\linewidth}
    \begin{center}
        \includegraphics[width=0.5\linewidth]{example-image}
        \vspace{-8pt}
    \end{center}
\end{minipage}
\begin{minipage}{0.45\linewidth}
    This is a little example text on the right side.\\
    \texttt{Could also be written} \textit{with different} \textbf{font styles}
\end{minipage}

\subsubsection{cite}
Hello World!\cite{latex2e}