\part{My First Part}

\chapter{My First Chapter}

\section{Section}



\subsection{Subsection}


\subsubsection{Subsubsection}
Lorem ipsum dolor sit amet, consectetur adipiscing elit, sed do eiusmod tempor incididunt ut labore et dolore magna aliqua. Ut enim ad minim veniam, quis nostrud exercitation ullamco laboris nisi ut aliquip ex ea commodo consequat. Duis aute irure dolor in reprehenderit in voluptate velit esse cillum dolore eu fugiat nulla pariatur. Excepteur sint occaecat cupidatat non proident, sunt in culpa qui officia deserunt mollit anim id est laborum.



\subsubsection{graphic}
\begin{figure}[H]
    \centering
    \includegraphics[scale=0.5]{img/OST_Icon}
    \caption{The OST Logo!}
\end{figure}

\subsubsection{table}
\begin{table}[H]
    \centering
    \caption{example tabular}
    \label{tab:example tabular}
        \begin{tabular}{l | c | r}
            \hline
            left align & center align & right align\\ \hline
            \hline
            hello & world & !\\ \hline
            hello & world & !\\ \hline
        \end{tabular}
\end{table}

\subsubsection{items}
\begin{itemize}
    \item hello
    \item world
    \item !
\end{itemize}

\subsubsection{enumeration}
\begin{enumerate}
    \item hello
    \item world
    \item !
\end{enumerate}

\subsubsection{listings}
\begin{lstlisting}[caption=example listing, style=bash]
# simple listing in default language
echo "Hello World"

cat /etc/resolv.conf
\end{lstlisting}

\begin{lstlisting}[firstnumber=17, caption=example listing, style=JavaScript]
// example listing in JS
    var = hello world
\end{lstlisting}


\subsubsection{minipage}
\begin{minipage}{0.5\linewidth}
    \begin{center}
        \includegraphics[width=0.5\linewidth]{example-image}
        \vspace{-8pt}
    \end{center}
\end{minipage}
\begin{minipage}{0.45\linewidth}
    This is a little example text on the right side.\\
    \texttt{Could also be written} \textit{with different} \textbf{font styles}
\end{minipage}

\subsubsection{cite}
Hello World!\cite{martinez-github}

\subsubsection{Glossary}
You can use short form \acrshort{gh} or long form \acrlong{gh} or full form \acrfull{gh}, all forms can also be used in plural \acrlongpl{gh}.
You can also use Glossary by itself, however it is better to use acronyms with glossary \Gls{github}.
All references can be clicked and will be redirected to glossary.